% !Mode:: "TeX:UTF-8"
%% 请使用 XeLaTeX 编译本文.
% \documentclass{WHUBachelor}% 选项 forprint: 交付打印时添加, 避免彩色链接字迹打印偏淡. 即使用下一行:
  \documentclass[forprint]{WHUBachelor}

  \begin{document}
  %%%%%%% 下面的内容, 据实填空.
  
  \miji{ 2016级}                                      % 密级. 没有就空着.
  \StudentNumber{16337259(谢江钊)} % 填写自己的学号
  
  \title{基于流水线CPU的分支预测技术性能研究}
  \Etitle{Research on the Performance of Branch Prediction Technology Based on Pipeline CPU} % 英文题目
  \author{谢江钊\quad 徐正金}                            % 作者名字
  \Eauthor{Xie Jiangzhao\quad Xv Zhengjin}            %作者英文名
  \Csupervisor{李国桢\quad 教授}        %指导教师中文名、职称
  \Esupervisor{Prof.~Li Guozhen}     %指导教师英文名、职称
  \Cmajor{计算机类}                  % 专业中文名
  \Emajor{Computer Class}% 专业英文名
  \Cschoolname{数据科学与计算机学院}          % 学院名
  \Eschoolname{School of Data and Computer Science} %学院英文名. 不确定的话, 请看一下自己学院的网页上是怎么写的. 别搞错了!
  \date{二〇一八年一月}                    % 日期, 要注意和英文日期一致!!
  \Edate{January, 2018}                       % 英文封面日期
  
  %-----------------------------------------------------------------------------
  \pdfbookmark[0]{封面}{title}         % 封面页加到 pdf 书签
  \maketitle
  \frontmatter
  \pagenumbering{Roman}              % 正文之前的页码用大写罗马字母编号.
  %-----------------------------------------------------------------------------
  % !Mode:: "TeX:UTF-8"

%%% 此部分需要自行填写: (1) 中文摘要及关键词 (2) 英文摘要及关键词
%%%%%%%%%%%%%%%%%%%%%%%%%%%%%
%%% -------------  英文封面 (无需改动)-------------   %%%
%%%%%%%%%%%%%%%%%%%%%%%%%%%%%
\thispagestyle{empty}
\renewcommand{\baselinestretch}{1.5}  %下文的行距
\vspace*{0.5cm}
\begin{center}
{\Large \bf Computer Organization Theory \\[1ex] Course Essay }
\end{center}
\vspace{2.5cm}
\begin{center}{\zihao{2} \the\Etitle \par}\end{center}

\vfill

\begin{center}
\zihao{4}
\begin{tabular}{ r l }
 School (Department): & {\sc \the\Eschoolname}\\
  Major:          &   {\sc\the\Emajor}  \\
 Candidate:      &  {\sc \the\Eauthor}      \\
 Supervisor:     &  {\sc \the\Esupervisor}
\end{tabular}

\vspace*{2cm}
\begin{center}
   \ifprint % 文档打印, 使用黑白校徽.
  \includegraphics[height=4cm]{sysu.pdf}       %%  黑白的.
  \else
  \includegraphics[height=4cm]{sysu.pdf} %%  彩色的.
  \fi
\end{center}


\zihao{-2}
%\the\Schoolname\\
{\sc Sun Yat-sun University}

\vspace*{1.0cm}

\the\Edate

\end{center}
%%% 郑重声明部分无需改动

%%%---- 郑重声明 (无需改动)------------------------------------%
\newpage
\vspace*{20pt}
\begin{center}{\ziju{0.8}\textbf{\songti\zihao{2} 郑重声明}}\end{center}
\par\vspace*{30pt}
\renewcommand{\baselinestretch}{2}

{\zihao{4}%

本人呈交的论文, 是独立进行研究工作所取得的成果,
所有数据、图片资料真实可靠. 尽我所知, 除文中已经注明引用的内容外,
本论文的研究成果不包含他人享有著作权的内容.
对本论文所涉及的研究工作做出贡献的其他个人和集体,
均已在文中以明确的方式标明. 本论文的知识产权归属于培养单位.\\[2cm]

\hspace*{1cm}本人签名: $\underline{\hspace{3.5cm}}$
\hspace{2cm}日期: $\underline{\hspace{3.5cm}}$\hfill\par}
%------------------------------------------------------------------------------
\baselineskip=23pt  % 正文行距为 23 磅
%------------------------------------------------------------------------------





%%======中文摘要===========================%
\begin{cnabstract}
在文章中我对CPU发展历史以来所采取的预测策略进行了软件上的模拟,对流水线CPU动态预测的性能作出了分析统计。关于此次的CPU设计,我实现了32位MIPS指令集。在学习CPU设计过程中,
流水线CPU的出现极大地提高了CPU的执行效率,但同时也需要合理解决三大冲突。对于控制冲突而言,需要做到尽可能准确预测。近代CPU在
这一方面做分支预测已经非常准确,但是流水线深度的逐渐增加使得哪怕微小的提升也将极大影响CPU的性能。因此文中对过往和近年来热门的预测方法作出评价。\\
为了减少工作量,我采用Python作为工具来解决这个问题。软件层面上我实现了对MIPS汇编程序的模拟执行,并对预测效果作出了统计。\\
分析结果表现混合分支预测器的性能非常优越,相比于传统的局部分支预测和全局分支预测,能够在更短时间内提高预测率并且拥有更加稳定准确的预测正确率。\\
文中的分析结果提供了一种可行性高的办法来评价CPU预测采取策略,可以作为探究分支策略效果的一种手段。


\end{cnabstract}
\par
\vspace*{2em}


%%%%--  关键词 -----------------------------------------%%%%%%%%
%%%%-- 注意: 每个关键词之间用“;”分开,最后一个关键词不打标点符号
\cnkeywords{计算机组成原理; 流水线CPU; 动态分支预测;  }


%%====英文摘要==========================%


\begin{enabstract}
  In the article, I have carried on the software simulation to the forecast tactics that has taken since the CPU development history, has carried on the analysis statistics to the pipeline CPU dynamic forecast performance. About this CPU design, I have realized 32 MIPS instruction set. In learning CPU design process,
  The advent of pipelined CPUs has greatly increased the efficiency of CPU implementation, but at the same time it also requires a reasonable solution to the three major conflicts. For control conflicts, forecasts need to be as accurate as possible. Modern CPU in
  Branch prediction is already very accurate on the one hand, but the gradual increase in pipeline depth makes even a small increase will greatly affect the CPU performance. Therefore, the text in the past and in recent years the popular prediction method to make an assessment. \\
  In order to reduce the workload, I use Python as a tool to solve this problem. At the software level, I implemented the simulation of the MIPS assembler and made statistics on the predicted results. \\
  The results of the analysis show that the performance of the hybrid branch predictor is superior. Compared with the traditional local branch prediction and global branch prediction, the hybrid branch predictor can improve the prediction rate in a shorter time and has a more stable and accurate prediction accuracy. \\
  The analysis results in this paper provide a highly feasible way to evaluate the CPU prediction strategy, which can be used as a means to explore the effect of the branch strategy.

\end{enabstract}
\par
\vspace*{2em}

%%%%%-- Key words --------------------------------------%%%%%%%
%%%%-- 注意: 每个关键词之间用“;”分开,最后一个关键词不打标点符号
 \enkeywords{Computer composition principle; Pipeline CPU; Dynamic branch prediction  }
    % 加入摘要, 申明.
  %==========================把目录加入到书签==============================%%%%%%
  \pdfbookmark[0]{目录}{toc}
  \tableofcontents
  \mainmatter %% 以下是正文
  %%%%%%%%%%%%%%%%%%%%%%%%%%%--------main matter-------%%%%%%%%%%%%%%%%%%%%%%%%%%%%%%%%%%%%
  \chapter{绪论}
   \section{分支预测算法的意义}
  
   分支预测器的重要性在课本中已经得到相当详尽的论述,在这里就不再从原理上去解释它了。回到工业生产中去,
近年来CPU的发展是巨大的,技术的更新集中在更高的制程,频率,深流水线,指令集并行,超线程技术等等上。其中
的流水线加深,更是使得条件分支带来的控制冲突变得日趋严重。分支预测进行得不准确,意味着预测的指令全部都
要抛弃不再进行,这一点会导致极大的性能浪费。其中,奔腾4便是一个失败的例子。奔腾4采用的31级超长流水线,
把主频提高到了3Ghz以上,但是性能却比不过同时期低主频14级流水线的AMD处理器,其中一个重要的原因便是没有
采用高效的分支预测器,导致流水线中断问题极其严重,导致了性能的严重下降。今天的CPU普遍采用14-16级流水线原因之一,
便是分支预测技术瓶颈。\par
业界引入了动态分支预测技术,开发出多种预测方案,使得控制冲突逐渐得到解决。其中较为贴近现代的便是局部分支
预测器,全局分支预测器,循环分支预测器以及间接分支预测器等等,在这里我将对一部分方案作出解释。另外,由
S. McFarling提出的混合(融合)分支预测器相比之下会有更加好的预测结果,同样非常值得考虑作为CPU预测器
发展的方向。\par
更前沿地,当前高端CPU已经迈进了神经网络预测以及多种预测器相结合的道路,在这里由于技术过于复杂便不再进行赘述。
预测器的发展需要考虑到速度和效果,这一点也是需要企业来进行权衡的。如何平衡两者的关系,也是未来发展的一个重点。
做这个项目的时候感触良多,对指令执行特点和CPU自身设计有了很多的体会。而这一个项目实现的内容将来也能够不断改进,
作为未来的教学预测性能预测手段。
  \section{技术发展现状与趋势}\label{sec-compile}

  \subsection{静态分支预测技术}
静态分支技术是最简单的技术,分为两种,强制跳转或强制不跳,而不依赖于过去的历史记录。后来的技术变得更加复杂,假定了
向后分支必然发生,而向前的分支不会发生,对应着程序设计中经常使用到的for循环和if判断特性,能够更好地进行预测。在
早期的时候一些处理器甚至允许分支预测提示出现在代码中,但是后来便不再采用。对于这种做法,理论上是不应该存在的,对于
程序员来说,底层的CPU实现应该做到透明,而不是反过来去负责做分支预测。\par
静态分支预测技术的优点在于对硬件实现基本没有要求,但是预测的效果却见不得理想,尤其对于日渐复杂的程序而言,其准确率
是不能够被接受的。


  \subsection{1bit分支预测器}
1bit分支预测器引入了一位分支预测缓存,记录分支最后一次的选择,然后预测下一次也做出同样的决定。但是这种预测方法有一个
问题在于,如果遇到TFTFTFT...的情况,会造成大量的空翻,从而使得性能被严重浪费。并且由于其总是预测分支跳转/不跳转一定
发生,一旦出现例外情况就会造成两次错误。为了解决这个问题,引入2bit分支预测器,来给预测引入一个缓冲区域。

  
  
  
  \subsection{2bit分支预测器}
2bit分支预测器是基于饱和计数器进行的,饱和计数器是一个拥有4个状态的状态机:强跳转,弱跳转,弱不跳转,他们之间的相互转换关系如下:\par
这种方法的优点在于指令必须选择某个分支连续两次,才能够实现状态的翻转,进而改变预测的分支。回到刚才的情况,一开始的时候
饱和计数器处于强跳转的状态,那么一次F的影响只能使计数器回到弱跳转状态,避免了接连的空翻,相比之下错误率下降。同样的道理,
我们还可以提出nbit分支预测器,来给予更多的缓冲空间,但是这种做法也导致了跳转不能够灵活地得到改变,在测试中反而导致预测率
下降。

 \subsection{局部分支预测器}
 前面所描述的2bit分支预测器,用状态机的办法把过去若干条指令改变的结果保存了下来,但是这样有一个问题,认为本次的跳转与
 之前的若干条指令跳转有关,但是处理起来却把本次预测结果应该基于过去的情况。举个例子,假如之前的几次跳转结果是TTF,那么
 这一次的结果如果仅仅看作前几次的衍生的话,便有可能出错。例如对于一个循环而言,循环3次跳转最后一次不跳转(跳出循环),如果
 依靠2bit分支预测器就会出现预测错误的结果,因为前三次的跳转不能够表明这一次也一定跳转。相对应的是,我们可以把前三次的历史
 记录看作是一种模式,然后在此基础上构建饱和计数器,经过若干次训练,便可以得到一个结果:三次的跳转必然伴随着一次结果,这样我们
 往后的预测便会变得准确,提高了我们的正确率。\par
 局部分支预测器构建了两个表:一个被称为BHT(Branch History Table),另一个被称为BPT(Branch Pattern Table)。
  
  
  \chapter{杂七杂八的话}
  
  \section{Readme}
  
  模板文件的结构, 如下表所示:
   \begin{table}[ht]\centering
  \begin{tabular}{r|r|l}
    \hline\hline
    \multicolumn{2}{l|}{Bachelor-template.tex }       & 主文档. 在其中填写正文.             \\ \hline
                                    & frontmatter.tex & 郑重声明、中英文摘要.               \\ \cline{2-3}
    \raisebox{1em}{includefile 文件夹} &  backmatter.tex & 致谢.                       \\ \hline
    \multicolumn{2}{l|}{figures 文件夹}                  & 存放图片文件.                   \\ \hline
    \multicolumn{2}{l|}{WHUBachelor.cls }             & 定义文档格式的 class file. 不可删除. \\ \hline\hline
  \end{tabular}
  \end{table}
  
  无需也不要改变、移动上述文档的位置.
  
  如果不习惯用~\verb|\include{ }|~的方式加入``子文档'', 当然可以把它们合并在主文档, 成为一个文档.
  ({\kaishu 但是这样并不会给我们带来方便.})
  
  利用~WinEdt~的~Project tree, 可以方便地管理这些文件:
  \begin{itemize}
      \item 点击~WinEdt~窗口的~Project Tree~按键;
      \item 再点击~WinEdt~窗口的~Set Main File~按键;
  \end{itemize}
  接下来的管理, 已经清楚地展示在跳出的窗口中了. 再去处理其他的文件时, 还要点击~WinEdt~窗口的~Remove Main File~按键.
  
  
  \section{更新记录}
  2016 年 06 月更新: 正文字体为小四号; 英文字体为 Times New Roman; 修订图表标题的字体、字号; 修订目录的字号; 修订附录章节编号的问题. 
                            非常感谢武汉大学数学与统计学院 2012 级张仕俊、林颖倩、宋俍辰等同学. 
  
  2016 年 05 月更新: 参考文献加到目录. 感谢武汉大学经济与管理学院的郑中天同学. [上次修订使用的版本有误, 非常抱歉.]
  
  2016 年 02 月更新: 调整为适应 TeX Live 2015 的版本.
  
  2014 年 06 月更新: 修改章节标题、声明标题、图表标题的字体和大小. 再次感谢孙启航同学.
  
  2014 年 05 月更新: 参考文献加到目录. 感谢武汉大学计算机学院孙启航同学、数学与统计学院李振坤同学指出这个纰漏.
  
  2013 年 12 月更新: 加上英文封面. 教务部的写作规范中的附例, 并没有英文封面. 但是遇到很多同学说要加上.
  
  
   \section{字体调节}
  
  \begin{tabular}{ll}
    \verb|\songti|   & {\songti 宋体}   \\
    \verb|\heiti|    & {\heiti 黑体}    \\
    \verb|\fangsong| & {\fangsong 仿宋} \\
    \verb|\kaishu|   & {\kaishu 楷书}
  \end{tabular}
  
  
  \section{字号调节}
  字号命令: \verb|\zihao| \index{zihao}
  
  \begin{tabular}{ll}
  \verb|\zihao{0}| &\zihao{0}  初号字 English \\
  \verb|\zihao{-0}|&\zihao{-0} 小初号 English \\
  \verb|\zihao{1} |&\zihao{1}  一号字 English \\
  \verb|\zihao{-1}|&\zihao{-1} 小一号 English \\
  \verb|\zihao{2} |&\zihao{2}  二号字 English \\
  \verb|\zihao{-2}|&\zihao{-2} 小二号 English \\
  \verb|\zihao{3} |&\zihao{3}  三号字 English \\
  \verb|\zihao{-3}|&\zihao{-3} 小三号 English \\
  \verb|\zihao{4} |&\zihao{4}  四号字 English \\
  \verb|\zihao{-4}|&\zihao{-4} 小四号 English \\
  \verb|\zihao{5} |&\zihao{5}  五号字 English \\
  \verb|\zihao{-5}|&\zihao{-5} 小五号 English \\
  \verb|\zihao{6} |&\zihao{6}  六号字 English \\
  \verb|\zihao{-6}|&\zihao{-6} 小六号 English \\
  \verb|\zihao{7} |&\zihao{7}  七号字 English \\
  \verb|\zihao{8} |&\zihao{8}  八号字 English \\
  \end{tabular}
  
  \section{已加入的常用宏包}
  
  \begin{description}
  %  \item[amsmath,amssymb]
    \item[cite]  参考文献引用, 得到形如 [3-7] 的样式.
    \item[color,xcolor]  支持彩色.
    \item[enumerate]  方便自由选择 enumerate 环境的编号方式. 比如
  
    \verb|\begin{enumerate}[(a)]| 得到形如 (a), (b), (c) 的编号.
  
  
    \verb|\begin{enumerate}[i)]| 得到形如 i), ii), iii) 的编号.
  
  \end{description}
  
  另外要说明的是,  itemize, enumerate, description 这三种 list 环境, 已经调节了其间距和缩进,
  以符合中文书写的习惯.
  
  \section{标点符号的问题}
  
  建议使用半角的标点符号, 后边再键入一个空格. 特别是在英文书写中要注意此问题!
  
  双引号是由两个左单引号、两个右单引号构成的: \verb|``  ''|. 左单引号在键盘上数字~1 的左边.
  
  但是, 无论您偏向于全角或半角, 强烈建议您使用实心的句号, 只要您书写的是自然科学的文章.
  原因可能是因为, 比如使用全角句号的句子结尾处的``$x$。''容易误为数学式~$x_0$(\verb|$x_0$|)吧.
  
  
  
  \section{引用的问题}
  
  
  \subsection{参考文献的引用}
  
  参考文献的引用, 用命令~\verb|\cite{ }|. 大括号内要填入的字串, 是自命名的文献条目名.
  
  比如, 通常我们会说:
  
   {\kaishu
  关于此问题, 请参见文献 \cite{r2}. 作者某某还提到了某某概念\upcite{r1}.}
  
  
  上文使用的源文件为:
  
   {\kaishu
  关于此问题, 请参见文献~\verb|\cite{r2}|. 作者某某还提到了某某概念~\verb|\upcite{r1}|.
  }
  
  其中~\verb|\upcite| 是自定义命令, 使文献引用呈现为\CJKunderdot{上标形式}.
  
  ({\heiti 注意:} {\kaishu 这里文献的引用, 有时需要以上标形式出现, 有时需要作为正文文字出现, 为什么?})
  
  另外, 要得到形如~\cite{r1,r3,r4,r5} 的参考文献连续引用, 需要用到 cite 宏包(模板已经加入),
  在正文中使用~\verb|\cite{r1,r3,r4,r5}| 的引用形式即可.
  或者, 连续引用的上标形式: 使用~\verb|\upcite{r1,r2,r3}|, 得到\upcite{r1,r2,r3}.
  
  \subsection{定理和公式的引用}
  
  \begin{theorem}[谁发现的]\label{th-abcd}
  最大的正整数是~$1$.
  \end{theorem}
  
  \begin{proof}
  要找到这个最大的正整数, 我们设最大的正整数为~$x$, 则~$x \geqslant 1$, 两边同时乘以~$x$, 得到
  \begin{equation}\label{eq-abc}
  x^2 \geqslant x.
  \end{equation}
  而~$x$ 是最大的正整数, 由~\eqref{eq-abc} 式得到
  \[
  x^2 = x.
  \]
  所以
  \begin{equation*}
  x = 1.
  \end{equation*}
  \end{proof}
  
  定理~\ref{th-abcd} 是一个重大的发现.
  
  %%%%----- 定义等环境的举例 --------
  \begin{definition}[整数]
   正整数(例如 1, 2, 3)、负整数(例如 ${−1}$, $−2$, $−3$)与零(0)合起来统称为{\heiti 整数}.
  \end{definition}
  
  \begin{remark}
    整数集合在数学上通常表示为 $\mathbf{Z}$ 或 $\mathbb{Z}$, 该记号源于德语单词 Zahlen(意为``数'')的首字母.
  \end{remark}
  
  \begin{proposition}
  任意两个整数相加、相减、相乘的结果, 仍然是整数.
  \end{proposition}
  
  \begin{example}
    $1+2=3$.
  \end{example}
  
  \begin{corollary}
     在整数集合内, 相加、相减、相乘运算是封闭的.
  \end{corollary}
  
  \section{图形与表格}
  
  支持对~eps, pdf, jpg 等等常见图形格式.
  
  再次\colorbox{red!45}{澄清一个误会}: \LaTeX{} 支持的图形格式绝非 eps 这一种. 无需特意把图片转化为 eps.
  
  用形如~\verb|\includegraphics[width=12cm]{Daisy.jpg}| 的命令可以纳入图片.
  
  如图~\ref{fig:1} 是一个纳入~jpg 图片的例子.
  
  \begin{figure}[ht]
  \centering
    \includegraphics[width=\textwidth]{Daisy.jpg}
    \caption{一个彩色 jpg 图片的例子}
    \label{fig:1}
  \end{figure}
  
  表格问题, 建议使用``三线表'', 如表 \ref{tab:1}.
  
  \begin{table}[ht]
  \centering
  \caption{一般三线表}
  \label{tab:1}
      \begin{tabular}{c c c c c c c c c c c}
      \hline
      123 & 4  & 5  & 123 & 4 & 5123 & 4 & 5 & 123 & 4 & 5\\
      \hline
      67 & 890 & 13 & 123 & 4 & 5123 & 4 & 5 & 123 & 4 & 5\\
      67 & 890 & 13 & 123 & 4 & 5123 & 4 & 5 & 123 & 4 & 5\\
      67 & 890 & 13 & 123 & 4 & 5123 & 4 & 5 & 123 & 4 & 5\\
      \hline
      \end{tabular}
  \end{table}
  
  
  
  
  
  
  %%%%============================================================================================================%%%
  
  \chapter{其他事项}
  以下是广告时间, 插播一段广告:
  \begin{itemize}
      \item 插图\index{插图}的制作, 建议用 pgf, 也叫 tikz.
            pgf 的长处是源文件直接植入~\TeX~文档, 管理起来非常方便.
      这里有我写的一个关于初次使用~pgf~的帖子:\\    \url{http://bbs.ctex.org/forum.php?mod=viewthread&tid=30480}.
      \item 生成参考文献, 建议使用~BibTeX.\index{BibTeX} 这里有我写的一个文档: \\
      \url{http://bbs.ctex.org/forum.php?mod=viewthread&tid=26056}.
  
            {\kaishu 使用 BibTeX{} 做参考文献时,
        借助 EndNote 或者 NoteExpress, 可以非常漂亮简单地解决 bib 文件的录入问题.
        NoteExpress 在校图书馆网站有正版软件提供下载.
        当然 EndNote 本身就是 Thomson Corporation 推出的(和 SCI 搜索引擎是同一家公司),
        和多个重要文献搜索引擎有良好的功能配合.
  
        Google 学术搜索也提供了文献的 bib 格式.
        录入参考文献时, 偶尔用一用 Google 学术搜索, 还可以核查或减少录入的错误, 并减少录入的工作量.}
  
      \item 幻灯片\index{幻灯片}的制作, 建议使用~Beamer. 这里有我写的一个模板, 谨供参考:\\
      \url{http://bbs.ctex.org/forum.php?mod=viewthread&tid=27695}.
  \end{itemize}
  
  
  %%%============================================================================================================%%%
  
  %%%=== 参考文献 ========%%%
  \cleardoublepage\phantomsection
  \addcontentsline{toc}{chapter}{参考文献}
  \begin{thebibliography}{00}
  
    \bibitem{r1} 作者. 文章题目 [J].  期刊名, 出版年份,卷号(期数): 起止页码.
  
    \bibitem{r2} 作者. 书名 [M]. 版次. 出版地:出版单位,出版年份:起止页码.
  
    \bibitem{r3} 邓建松等, 《\LaTeXe~科技排版指南》, 科学出版社.
  
    \bibitem{r4} 吴凌云, 《CTeX~FAQ (常见问题集)》, \textit{Version~0.4}, June 21, 2004.
  
    \bibitem{r5} Herbert Vo\ss, Mathmode, \url{http://www.tex.ac.uk/ctan/info/math/voss/mathmode/Mathmode.pdf}.
  
  
  \end{thebibliography}
  
  % !Mode:: "TeX:UTF-8"
%%%%%%%%%%%%%%%%%%%%%%%%%%%%-------致谢--------%%%%%%%%%%%%%%%%%%%%%%%%%%%%%%%%

\acknowledgement
\addcontentsline{toc}{chapter}{致谢}


作为计算机组成原理最后的一份大作业,在这里要对李国桢老师致以真诚的感谢。无论是课堂亦或者是解答问题对我们都非常有耐心。在开展最后的大作业的过程中,我通过查资料,找论文来理解预测方法,
并且实现目标,发现了很多自己以前其实并不是怎么了解的概念,增强了自己获取资料,自主学习的能力,感觉收获匪浅。\par
最后,再对老师一学期以来的教导表示感谢。











 %%%致谢
  
  %%%-------------- 附录. 不需要可以删除.-----------
  \appendix
  
  \chapter{测试}
  
  \section{第一个测试}
  测试公式编号
  \begin{equation}
  1+1=2.
  \end{equation}
  
  表格编号测试
  
  \begin{table}[h]
    \centering
    \caption{测试表格}
    \begin{tabular}{*{20}c}
       \hline
       % after \\: \hline or \cline{col1-col2} \cline{col3-col4} ...
       11 & 13  & 13  & 13  & 13 \\
       12 & 14  & 13  & 13  & 13 \\
       \hline
     \end{tabular}
  \end{table}
  
  
  \chapter{附录测试}
  
  测试
  
  \chapter{附录测试}
  
  测试
  
  \cleardoublepage
  \end{document}
  
  
  
  